% tex code for report
\documentclass{article}
\usepackage{url}
\usepackage[margin=0.75in]{geometry}
\usepackage{cite}
\usepackage{amssymb}

  \title{Comp3260 Assignment 1}
  \author{Jay Rovacsek and Cody Lewis}
  \date{\today}
  
  \begin{document}
  
    \maketitle
    \section{c1 Ciphertext}
    c1 was found to be a Viginere cipher with the key: `remember'.
      \subsection{Analysis}
      First the frequency graph was observed, which expressed that the cipher
      was produced by substitution. Next the IC was calculated giving an 
      approximate period, d = 3. Then a Kasiski of 17 characters shown that the
      string `agxyvyzmffhjkefie' occurs in the text 3 times with the gaps
      392 and 2024.
        \[ GCD(2024,392): 2024 = 5 \times 392 + 64 \] 
        \[ GCD(392,64): 392 = 6 \times 64 + 8 \] 
        \[ GCD(64,8): 64 = 8^2 + 0 \] 
        \[ \Rightarrow GCD(2024,392) = 8 \] 
      Hence the cipher definitely has 8 alphabets. Looking at the frequency 
      graph with 8 alphabets shows that the cipher is Viginere as the letter 
      frequency match the normal graph left to right, they only shift. The 
      key is the letter a is shifted to in each alphabet, which means the key 
      is `remember'.
    \section{c2 Ciphertext}
    c2 was found to be a Beauford cipher with the key: `triumph'.
      \subsection{Analysis}
      The frequency graphs shows that a substitution cipher was used, the IC
      give an approximate period, d = 10. A Kasiski at 10 gives the gaps of 287,
      91.
        \[ GCD(287,91) = 7 \]
      Hence there are definitely 7 alphabets in the cipher. Solving for the key
      to this cipher is very similar to the previous one, only the frequency 
      graph is backwards meaning it is a Beauford cipher. Matching each letter
      to the letter in place of a in each alphabet, gives the key `triumph'.
    \section{c3 Ciphertext}
      c3 was found to be a General substitution cipher with the key: `mcafrgyjhkdplqiutvexbzwsno'
      \subsection{Analysis}
      General substitution was determined by the IC being analysed over 1\-26 alphabets 
      and returning an average of 0.06836808. Leading to the assumption that only a single
      alphabet had been used.\\
      Furthermore, flags making c3 seem like a substitution cipher were the high counts of strings
      between 3 \- 4 letters which stood out very abnormally from the remaining distribution.\\
      \subsection{Method of breaking} 
      A boon to the breaking of the substitution cipher was Simon Singh's page\cite{SimonSinghNetTBC}
      on cracking the substitution cipher and likely trigrams and digrams to occur within the English language.
      Moving forward with the assumption that the most common letters of the ciphertext were likely
      to mimic that of English, it was determined that: \[the = xjr\] \[and = hqy\]
      However after a number of hours attempting to find words containing a mixture of the two
      strings, it was obvious that `and' did not match the assumed ciphertext and instead another
      common three letter string that had occurred. It was still assumed `the' matched the ciphertext proposed
      as the distribution was skewed such that it would be highly unlikely `t' could have been 
      anything else.\\
      Eventually a key was built based on comparison of ciphertext elements that were assumed to be certain words
      and if incorrect another key:value was attempted instead. \\
      \subsection{Second Analysis}
      On finding that \[ the = xjr \] the cipher was checked as to whether it was an affine transformation. The 
      following equations were formed:
      \begin{equation}
        e \mapsto r \Rightarrow (4 k_0 + k_1) \pmod{26} = 17
      \end{equation}
      \begin{equation}
        h \mapsto j \Rightarrow (7 k_0 + k_1) \pmod{26} = 9
      \end{equation}
      \begin{equation}
        t \mapsto x \Rightarrow (19 k_0 + k_1) \pmod{26} = 23
       \end{equation}
       Using Gaussian elimination for (1) \- (2) brings the equation:
       \begin{equation}
         -3 k_0 \pmod{26} = 8 \Rightarrow 23 k_0 \pmod{26} = 8
       \end{equation}
       \[ GCD(26,23): 26 = 1 \times 23 + 3 \]
       \[ GCD(23,3):  23 = 7 \times 3 + 2 \] 
       \[ GCD(3,2): 3 = 2 \times 1 + 1 \] 
       \[ \Rightarrow 1 = 3 - 2 \] 
       \[ \Rightarrow 1 = 3 - (23 - 7 \times 3) \] 
       \[ \Rightarrow 1 = 8 \times 3 - 23 \] 
       \[ \Rightarrow 1 = 8(26 - 23) - 23 = 8 \times 26 - 9 \times 23 \]
       \begin{equation}
         \therefore k_0 = [-9] \times 8 \pmod{26} = 6
       \end{equation}
       Subbing (4) into (3) brings:
       \begin{equation}
         k_1 = 23 - (19 \times 6 \pmod{26}) = 13
       \end{equation}
       But these concluded values do not work for equations (1) and (2), hence c3 is not an affine transformation.
      \section{c4}
      c4's IC indicated that it is monoalphabetic, and its frequency graph matches that
      of normal English, hence c4 is a Transposition cipher. The first 3 letters of the
      cipher text is `hte' according to digraphs is probably the message `the', thus the
      cipher is a row Transposition. Through some anagramming further through the message 
      the words `enigma machine was' were found. After a bit of evaluation of the pattern 
      against ciphertext and message the key `2,1,4,6,5,4' or `bafced' was found to decipher 
      c4 into the message. \\

      \begin{thebibliography}{9}
        \bibitem{SimonSinghNetTBC}
          Simon Singh,
          \textit{The Black Chamber \- Hints and tips},
          \url{http://simonsingh.net/The_Black_Chamber/hintsandtips.html}

        \bibitem{jkypto}
          Lawrie Brown based on code by Daryl Bossert,
          \textit{jkypto}
          The program used for most of the statistical analysis, decryption and Kasiski calculations.

      \end{thebibliography}
  \end{document}  
