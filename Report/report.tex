% tex code for report
\documentclass{article}

\title{Comp3260 Assignment 1}
\author{Jay Rovacsek and Cody Lewis}
\date{\today}

\begin{document}

  \maketitle

  \section{c1}
    First the frequency graph was observed, which expressed that the cipher
    war produced by substitution. Next the IC was calculated giving an 
    approximate period, d = 3. Then a kasiski of 17 characters shown that the
    string `agxyvyzmffhjkefie' occurs in the text 3 times with the gaps
    392 and 2024.
      \[ GCD(2024,392): 2024 = 5 \times 392 + 64 \] 
      \[ GCD(392,64): 392 = 6 \times 64 + 8 \] 
      \[ GCD(64,8): 64 = 8^2 + 0 \] 
      \[ \Rightarrow GCD(2024,392) = 8 \] 
    Hence the cipher definitely has 8 alphabets. Looking at the frequency 
    graph with 8 alphabets shows that the cipher is Viginere as the letter 
    frequency match the normal graph left to right, they only shift. The 
    key is the letter a is shifted to in each alphabet, which means the key 
    is `remember'.
  \section{c2}
    The frequency graphs shows that a substitution cipher was used, the IC
    give an approximate period, d = 10. A Kasiski at 10 gives the gaps of 287,
    91.
      \[ GCD(287,91) = 7 \]
    Hence there are definitely 7 alphabets in the cipher. Solving for the key
    to this cipher is very similar to the previous one, only the frequency 
    graph is backwards meaning it is a Beauford cipher. Matching each letter
    to the letter in place of a in each alphabet, gives the key `triumph'.
  \section{c3}
    c3 was found to be a General substitution cipher
  \section{c4}
    c4's IC indicated that it is monoalphabetic, and its frequency graph matches that
    of normal English, hence c4 is a Transposition cipher. The first 3 letters of the
    cipher text is `hte' according to digraphs is probably the message `the', thus the
    cipher is a row Transposition. Through some anagramming further through the message 
    the words `enigma machine was' were found. After a bit of evaluation of the pattern 
    against ciphertext and message the key `2,1,4,6,5,4' was found to decipher c4 into the
    message. \\
    c4 was found to be a Transposition cipher. The key is most certainly a length of 
    x mod (6) = 0 give the Kasiski analysis on c4 which resulted in results for the
    strings `ahcmni' and `cyretp' which when compared against first seen position and
    final positions ended having a GCD of 6.
      \[ GCD(492,414) = 6 \]
\end{document}
